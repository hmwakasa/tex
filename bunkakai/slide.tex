%! LuaLaTeX 文書
\documentclass[unicode,12pt,colorlinks,]{beamer}
\usetheme{hmnotebook} % https://github.com/kmaed/kmbeamer
\usefonttheme{professionalfonts}

\hypersetup{linkcolor=,urlcolor=teal}

\usepackage{luatexja}
\ltjsetparameter{jacharrange={-2,-3,-8}}
\usepackage[no-math,match,deluxe]{luatexja-preset}

\usepackage{textcomp}
\usepackage{luatexja-otf}

\usepackage{graphicx,xcolor}
\usepackage{pxrubrica}
\usepackage{tcolorbox}
\usepackage{bxwareki}
\usepackage{indentfirst}
\usepackage{hmtrump}

\usepackage{listings}
\lstset{
	doubleletterspace,
	basicstyle={\small\ttfamily},
	backgroundcolor={\color[gray]{.9}},
	commentstyle={\color[gray]{.5}},
	numbers=left,
	numberstyle=\tiny,
}

\usepackage{tikz}
\usetikzlibrary{plotmarks}

\usepackage{scsnowman} %☃
\usepackage{bxokumacro}
\usepackage{keystroke}
\usepackage{bxrawstr} % https://zrbabbler.hatenablog.com/entry/20181222/1545495849
\usepackage[twemoji-pdf]{bxcoloremoji} % https://github.com/zr-tex8r/BXcoloremoji

\usepackage{mflogo}

\usepackage{bxtexlogo}
\bxtexlogoimport{*,**}

\usepackage[T1]{fontenc}
\usepackage{mathtools,amssymb,mathrsfs,mathabx}
\usepackage[math]{iwona}
\usepackage[euler-digits]{eulervm}
\allowdisplaybreaks[4]

%%%%%%%%%% 商用フォントを用いているので、自分でタイプセットする場合は適当に以下を変えてください
\setmainfont[
	Ligatures=TeX,
	BoldFont=FOT-KafuMarkerStd-B,
	ItalicFont=FOT-KafuMarkerStd-B,
]{FOT-KafuPenjiStd-M}
\setsansfont[
	Ligatures=TeX,
	BoldFont=FOT-KafuMarkerStd-B,
	ItalicFont=FOT-KafuMarkerStd-B,
]{FOT-KafuPenjiStd-M}
\setmainjfont[
	Ligatures=TeX,
	CharacterWidth=Proportional,
	JFM=prop,
	BoldFont=FOT-KafuMarkerStd-B,
	ItalicFont=FOT-KafuMarkerStd-B,
]{FOT-KafuPenjiStd-M}
\setsansjfont[
	Ligatures=TeX,
	CharacterWidth=Proportional,
	JFM=prop,
	BoldFont=FOT-KafuMarkerStd-B,
	ItalicFont=FOT-KafuMarkerStd-B,
]{FOT-KafuPenjiStd-M}
\setmonofont[
	Ligatures=TeX,
	Scale=0.89,
]{DejaVu Sans Mono}
\setmonojfont[
	Ligatures=TeX,
	Scale=0.89,
]{NotoSansMonoCJKjp-Regular}

\newfontfamily\errorfont[
	Ligatures=TeX,
]{TsukuQMinLStd-L}

\newfontfamily\nishikifonta[
	Ligatures=TeX,
]{Nishiki-teki}
\newjfontfamily\nishikifontj[
	Ligatures=TeX,
]{Nishiki-teki}
\newcommand{\nishikifont}{\nishikifonta\nishikifontj}

%%%%%%%%%% 以下は自前コマンド

\newcommand{\centeralign}[1]{\rule{0pt}{0pt}\hfill#1\hfill\rule{0pt}{0pt}}
\newcommand{\typecommand}[1]{\colorbox{darkgray}{{\ttfamily\color{lime}#1}}}

\setlength{\parskip}{2ex}

\title{電脳スタイルを作り変えている話}
\subtitle{Dennou6.sty から dennou777.cls へ}
\author{北海道大学理学部 ひとみさん(人見祥磨)}
\date{\warekitoday}

\begin{document}

\begin{frame}
	\maketitle
	
	\centeralign{{\footnotesize 発表資料: \url{http://www.circle9.work/tex/bunkakai/}}}
\end{frame}

\begin{frame}
	\frametitle{自己紹介}
	\begin{itemize}
		\item 北海道大学 GFD 研究室 B4
		\item CTAN にパッケージをひとつ上げている
			\begin{itemize}
				\item \texttt{hmtrump} パッケージ
				\item \url{https://www.ctan.org/pkg/hmtrump}
				\item トランプのカードを表現するためのパッケージ \trump AS
				\item \TeX\ Live に入ってます
			\end{itemize}
		\item epNetFan で \TeX 入門を講演
			\begin{itemize}
				\item \url{http://www.circle9.work/tex/eptex/}
			\end{itemize}
		\item \TeX 自体は 4 年前から使っている
	\end{itemize}
\end{frame}


\begin{frame}
	\frametitle{やってること}
	電脳スタイル \texttt{dennou6.sty} の改修

	\begin{itemize}
		\item バグの原因を修正
		\item 非推奨なコマンドの除去
		\item ニュー・スタンダードを取り入れる
	\end{itemize}

	中身がかなり書き換わる→新しく開発
\end{frame}

\begin{frame}
	\begin{center}
		{\LARGE\aruby{\texttt{dennou777.cls}}{電脳スリーセブン}}

		\url{https://github.com/Hitomi-San/dennou777}

		最新バージョン: 2019-09-03 version 7.0.6 (Nicole)

		\vspace{5ex}
		ぜひ皆さんも利用して、issue など投げてください
	\end{center}
\end{frame}

\begin{frame}[fragile]
	\frametitle{動機}
	\begin{block}{どうして開発しようと思ったのか}
		\begin{itemize}
			\item \texttt{dennou6.sty} の最終更新が 2001 年
				\begin{itemize}
				\item 2001 年から、今までにあった \TeX の出来事
					\begin{itemize}
						\item \pTeX が \TeX\ Live に収録された
						\item 日本語 \TeX 開発コミュニティの登場
						\item \LuaTeX など、新しいドライバの登場
					\end{itemize}
				\end{itemize}
			\item 非推奨なコマンドが多用されている(i.e.\ \verb|\bf|)
		\end{itemize}

		\texttt{dennou6.sty} が時代に追いついていない\\
		(逆に古すぎて有害)
	\end{block}
\end{frame}

\begin{frame}
	\frametitle{動機}
	\begin{block}{どうしてクラスファイル (\texttt{*.cls}) として開発するのか}
		\begin{itemize}
			\item 電脳スタイルの性格上、クラスファイルが最適と判断
				\begin{itemize}
					\item \texttt{D6style.sty} 相当の機能
					\item ページスタイルの定義
					\item 見出しの定義
				\end{itemize}
			\item \texttt{jナントカ.cls} と \texttt{Dennou6.sty} の組み合わせを排除したい
				\begin{itemize}
					\item \texttt{jナントカ.cls} は非推奨
					\item \texttt{jsナントカ.cls} のほうが良い
					\item \texttt{jlreq.cls} など新しいものも登場
				\end{itemize}
		\end{itemize}
	\end{block}
\end{frame}

\begin{frame}
	\frametitle{開発の方針}
	\begin{itemize}
		\item Version 6 と同様な機能を提供する
			\begin{itemize}
				\item 完全に再現することは保証しない
			\end{itemize}
		\item 非推奨な記述は改める
		\item \emph{\pLaTeX + dvipdfmx 以外の方法もサポートする}
			\begin{itemize}
				\item \upLaTeX + dvipdfmx や \LuaLaTeX-ja など
				\item \XeLaTeX は日本語するための土台が不十分なので対象外
			\end{itemize}
	\end{itemize}
\end{frame}

\begin{frame}[fragile]
	\frametitle{開発の方針}
	\begin{block}{jlreq をベースにする}
		\begin{itemize}
			\item \texttt{jlreq} は \pLaTeX, \upLaTeX, \LuaLaTeX に対応
			\item 版面構成のオプションが多い
			\item \verb|\LoadClassWithOptions{jlreq}|
		\end{itemize}
	\end{block}

	\begin{block}{D6style の機能は jlreq で実装}
		\begin{itemize}
			\item \verb|\NewPageStyle| の利用
			\item ヘッダーとフッターの罫線を出力するオプションがない
				\begin{itemize}
					\item \texttt{fancyhdr} の利用を試みたが……
					\item \verb|\NewPageStyle| したものが上書きされる
					\item とりあえず罫線は諦める
				\end{itemize}
			\item ヘッダーとフッターの書式は再現できた(はず)
		\end{itemize}
	\end{block}
\end{frame}

\begin{frame}[fragile]
	\frametitle{開発の現況}
	\begin{itemize}
		\item D6style 相当の機能は提供できている
		\item バグがいくつか残ってる(ドキュメントに記載)
		\item 未実装の機能は dennou6 のファイルを直接利用している
			\begin{itemize}
				\item \verb|\let\bf\bfseries| とかして誤魔化してる
				\item つぎは D6math に手を付けたい
			\end{itemize}
		\item jlreq の不具合・仕様変更が直撃するのをどうにかしたい
			\begin{itemize}
				\item \texttt{description} 環境の左余白とか
			\end{itemize}
	\end{itemize}
\end{frame}
\end{document}
